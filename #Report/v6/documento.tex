\documentclass{report}
\usepackage[T1]{fontenc} % Fontes T1
\usepackage[utf8]{inputenc} % Input UTF8
\usepackage[backend=biber, style=ieee]{biblatex} % para usar bibliografia
\usepackage{csquotes}
\usepackage[portuguese]{babel} %Usar língua portuguesa
\usepackage{blindtext} % Gerar texto automaticamente
\usepackage[printonlyused]{acronym}
\usepackage{hyperref} % para autoref
\usepackage{graphicx}
\usepackage{indentfirst}
\usepackage{textcomp}
\usepackage[skip=5pt]{caption}
%\usepackage[margin=75pt,footskip=40pt]{geometry} FICA TOPS

\bibliography{bibliografia}


\begin{document}
%%
% Definições
%
\def\titulo{Projeto Final de Laboratório de Sistemas Digitais}
\def\data{01 de Junho de 2023}
\def\autores{Olha Buts, André Correia}
\def\autorescontactos{(112920) o.buts@ua.pt, (87818) amcorreia@ua.pt}
\def\versao{VERSÃO 0.1}
\def\departamento{Dept. de Eletrónica, Telecomunicações e Informática}
\def\empresa{Universidade de Aveiro}
\def\logotipo{ua.pdf}
%
%%%%%% CAPA %%%%%%
%
\begin{titlepage}

\begin{center}
%
\vspace*{50mm}
%
{\Huge \titulo}\\ 
%
\vspace{10mm}
%
{\Large \empresa}\\
%
\vspace{10mm}
%
{\LARGE \autores}\\ 
%
\vspace{30mm}
%
\begin{figure}[h]
\center
\includegraphics{\logotipo}
\end{figure}
%
\vspace{30mm}
\end{center}
%
\begin{flushright}
\versao
\end{flushright}
\end{titlepage}

%%  Página de Título %%
\title{%
{\Huge\textbf{\titulo}}\\
{\Large \departamento\\ \empresa}
}
%
\author{%
    \autores \\
    \autorescontactos
}
%
\date{\today}
%
\maketitle

\pagenumbering{roman}

%%%%%% RESUMO %%%%%%
%\begin{abstract}
%popopopopopoqewp
%\end{abstract}

%%%%%% Agradecimentos %%%%%%
% Segundo glisc deveria aparecer após conclusão...
%\renewcommand{\abstractname}{Agradecimentos}
%\begin{abstract}
%Eventuais agradecimentos.
%Comentar bloco caso não existam agradecimentos a fazer.
%\end{abstract}


\tableofcontents
% \listoftables     % descomentar se necessário
\listoffigures    % descomentar se necessário


%%%%%%%%%%%%%%%%%%%%%%%%%%%%%%%
\clearpage
\pagenumbering{arabic}

%%%%%%%%%%%%%%%%%%%%%%%%%%%%%%%%
\chapter{Introdução}
\label{chap.introducao}
Os alunos da Unidade Curricular de \textbf{L}aboratórios de \textbf{S}istemas \textbf{D}igitais (\textit{LSD}, código 40333) da \textbf{L}icenciatura em \textbf{E}ngenharia de \textbf{C}omputadores e \textbf{I}nformática (\textit{LECI}, código 8316) da \textbf{U}niversidade de \textbf{A}veiro (\textit{UA}) foram propostos ao desenvolvimento de um Projeto Final que contempla três componentes: desenvolvimento do sistema digital de uma Máquina Automática de Fazer Pão (Projeto Número 8, Versão 2), criação de um relatório do desenvolvimento anteriormente referido, e defesa do projeto perante um Juri.
\\\\
O sistema digital da Máquina Automática de Fazer Pão deve ser modelado em \textit{\textbf{V}ery High Speed Integrated Circuits \textbf{H}ardware \textbf{D}escription \textbf{L}anguage} (\textit{VHSIC-HDL}, ou \textit{VHDL}) e testado numa \textbf{F}ield-\textbf{P}rogrammable \textbf{G}ate \textbf{A}rray (\textit{FPGA}). Neste sentido, a máquina desenvolvida apresenta dois modos de operação principal: Fazer Pão Caseiro (Modo 1), ou Fazer Pão Rústico (Modo 2) -- sendo que ambos os modos apresentam a possibilidade de adição de um tempo extra inicial, ou adição de um tempo extra final (que aumenta o tempo total de cozedura do pão). Apesar de cada um destes modos ser caracterizado por diferentes parâmetros temporáis, ambos partilham a mesma \textit{pipeline}, ou procedimento de 'fazer pão'  (o amassar da massa, o descanso da massa para levedar, e a cozedura no final).
\\\\
Relativamente ao documento, este apresenta o relatório do desenvolvimento do sistema digital da Máquina Automática de Fazer Pão (Versão 2 do Projeto 8) de acordo com as competências adquiridas na Unidade Curricular de LSD.
Neste sentido, o documento divide-se em quatro componentes, sendo estas a arquitetura do sistema digital (descrição conceptual do sistema), a implementação efetuada para a anterior arquitetura (representação gráfica do sistema digital), os métodos de validação usados (simulações efetuadas sobre a implementação da arquitetura), e por fim, um manual de utilizador da máquina como um todo (em ambiente de desenvolvimento através de uma FPGA).

\chapter{Desenvolvimento do Sistema Digital}
\label{chap.desenvolvimentoSistemaDigital}
O desenvolvimento e implementação do sistema digital desta máquina passa por três fases: arquitetura (desenho lógico de todo o funcionamento do sistema), implementação (em VHDL, usando o programa Intel{\textsuperscript{\tiny{\textregistered}}} Quartus{\textsuperscript{\tiny{\textregistered}}} Prime) e uma posterior validação (testes via \textit{testbenches} em VHDL e via uso normal, numa ótica de utilizador).


\section{Arquitetura do Sistema}
Nesta secção aborda-se a estrutura do sistema digital através de uma descrição conceptual da lógica que gerou o produto final (em ambiente de desenvolvimento por via de FPGA). Neste sentido, a arquitetura aplicada neste projeto divide-se em duas zonas que estão intrinsecamente interligadas:
\begin{itemize}
	\item Zona de controlo do sistema, responsável maioritariamente pelos \textit{inputs} -- por exemplo:
	\begin{itemize}
		\item Conjunto (físico / \textit{hardware}) de \textit{keys}.
		\item Conjunto de \textit{switches}.
		\item Comportamento (lógico) de \textit{start}/\textit{stop}.
		\item Comportamento de \textit{reset} do sistema.
	\end{itemize}
	\item Zona de controlo do procedimento de 'fazer pão', responsável pela maior parte do \textit{output}, e por toda a funcionalidade de 'fazer pão', -- por exemplo:
	\begin{itemize}
		\item Comportamento de cada etapa do processo de amassar, levedar e cozer o pão.
		\item Output do estado atual da máquina, assim como de informações ao utilizador, através de componentes físicos tais como o \textit{\textbf{L}iquid-\textbf{C}rystal \textbf{D}isplay} (LCD) e os \textit{7-Segment Displays} da FPGA.
	\end{itemize}
\end{itemize}
Estas duas zonas de controlo são ambas compostas por elementos lógicos \textit{standard}, assim como por controladores costumizados -- \textbf{M}áquinas de \textbf{E}stados \textbf{F}initos (MEF, ou FSM em Inglês). Estas FSM caracterizam-se por serem comunicantes, o que possibilita comportamentos que interligam simultâneamente a lógica das duas zonas de controlo -- destacando-se o caso da possibilidade de adicionar um tempo extra de cozedura (no final do processo de 'fazer pão').
\\\\
Na Figura 2.1 visualiza-se o diagrama lógico completo do sistema, estando a azul destacada a zona de controlo do sistema (denominada doravante por FSM de Controlo), e a verde destacada a zona de controlo do procedimento de 'fazer pão' (doravante denominada por FSM Principal).
\begin{figure}[h!] % Este [h!] força a imagem a aparecer onde é chamada.
	\center
	\includegraphics[width=335pt]{images/fase2v9_2}
	\caption{Diagrama Lógico do Sistema.}
	\label{fig:imagem1}
\end{figure}
\\
Assim, simplifica-se o circuito lógico através da abstração representada pela Figura 2.2 onde se destaca, conceptualmente, o sistema. Nesta Figura pode-se identificar que o sistema é constituído por duas FSM principais que controlam logicamente o conjunto de \textit{inputs} e \textit{outputs} fornecidos pela FPGA. A FSM de Controlo pode ser descrita pela responsabilidade dos mecanismos de \textit{start}/\textit{stop}, \textit{reset}, temporizadores (ligados à lógica de adição de tempos extra no início, ou no final), e pela seleção do modo de programa (tipo de pão Caseiro, ou Rústico). A FSM Principal é reponsável por toda a \textit{pipeline} de 'fazer pão' -- isto é, os estados de amassar, levedar e cozer.
\begin{figure}[h!] % Este [h!] força a imagem a aparecer onde é chamada.
	\center
	\includegraphics[width=250pt]{images/SistemaAbstrato_2}
	\caption{Abstração do Diagrama Lógico do Sistema.}
	\label{fig:imagem2}
\end{figure}

\section{Implementação do Sistema}
Nesta secção ilustram-se alguns aspetos da implementação da arquitetura anteriormente apresentada, tais como os diagramas de estados das FSM de Controlo e FSM Principal, assim como uma apresentação (em lista) dos componentes \textit{standard} utilizados para que a máquina (como um todo) funcione como esperado.
\\\\
Deste modo, começa-se por analisar o diagrama de estados da FSM de Controlo (ver Figura 2.3) onde se destaca a responsabilidade de iniciar a \textit{pipeline} da FSM Principal, assim como a responsabilidade de pausar todo o procedimento -- seja no momento de amassar, levedar, ou amassar o pão. Ainda pela Figura 2.3 se denota o controlo dos tempos extras (inicial, ou final).
\begin{figure}[h!] % Este [h!] força a imagem a aparecer onde é chamada.
	\center
	\includegraphics[width=280pt]{images/FSM1_2}
	\caption{Diagrama de Estados da FSM de Controlo.}
	\label{fig:imagem3}
\end{figure}
\\
Relativamente à FSM Principal, apesar desta conter uma interface que alimenta alguns sinais de input da FSM de Controlo (por exemplo, o sinal "\textit{newPrg}" que indica que terminou o seu ciclo), denota-se uma responsabilidade quase total pela pipeline de amassar, levedar, e cozer o pão -- onde a diferença entre o pão caseiro e rústico é no tempo de amassar (10s/15s, respetivamente) e de levedar (04s/08s); cozer (10s/10s) -- sendo que os tempos específicos para cada tipo de pão (ou modo de operação) são alimentados por uma o sinal \textit{\textbf{R}ead \textbf{O}nly \textbf{M}emory} (ROM).
\begin{figure}[h!] % Este [h!] força a imagem a aparecer onde é chamada.
	\center
	\includegraphics[width=280pt]{images/FSM2_2}
	\caption{Diagrama de Estados da FSM Principal.}
	\label{fig:imagem4}
\end{figure}

\subsection{Utilização de acrónimos}
Esta é a primeira invocação do acrónimo \ac{ua}.
E esta é a segunda \ac{ua}.

\section{Validação do Sistema}

\subsection{Utilização de acrónimos}
Esta é a primeira invocação do acrónimo \ac{ua}.
E esta é a segunda \ac{ua}.

\chapter{Manual de Utilizador}
\label{chap.manualUtilizador}
Descreve os resultados obtidos.

\chapter{Conclusões}
\label{chap.conclusao}
A arquitetura da Máquina Automática de Fazer Pão e a posterior implementação através de Máquinas de Estados Finitos Comunicantes, e toda a envolvente lógica digital, demonstrou atingir um nível de complexidade que necessita de procedimentos de desenvolvimento bem definidos desde o início do projeto.
\\\\
Deste modo, é de importância realçar a necessidade de estratégias de desenvolvimento faseadas e de mecanismos de controlo, tais como versões de projeto, assim como metodologia em todas as etapas do projeto. Por se prever este nível de complexidade, esta equipa preparou um repositório (...)



\chapter*{Contribuições dos autores}
Resumir aqui o que cada autor fez no trabalho.
Usar abreviaturas para identificar os autores,
por exemplo AS para António Silva.

\vspace{10pt}
\textbf{Indicar a percentagem de contribuição de cada autor.}\\


\autores : 50\%, 50\%\\

%\textbf{Repositório GitHub:} labi2023gN

%%%%%%%%%%%%%%%%%%%%%%%%%%%%%%%%%
%\chapter*{Acrónimos}
%\begin{acronym}
%\acro{ua}[UA]{Universidade de Aveiro}
%\acro{leci}[LECI]{Licenciatura em Engenharia de Computadores e Informática}
%\acro{glisc}[GLISC]{Grey Literature International Steering Committee}
%\acro{lsd}[LSD]{Laboratórios de Sistemas Digitais}
%\acro{vhdl}[VHDL]{Very \textit{High Speed Integrated Circuits} Hardware Description Language}
%\end{acronym}


%%%%%%%%%%%%%%%%%%%%%%%%%%%%%%%%%
\printbibliography

\end{document}
