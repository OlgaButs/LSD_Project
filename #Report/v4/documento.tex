\documentclass{report}
\usepackage[T1]{fontenc} % Fontes T1
\usepackage[utf8]{inputenc} % Input UTF8
\usepackage[backend=biber, style=ieee]{biblatex} % para usar bibliografia
\usepackage{csquotes}
\usepackage[portuguese]{babel} %Usar língua portuguesa
\usepackage{blindtext} % Gerar texto automaticamente
\usepackage[printonlyused]{acronym}
\usepackage{hyperref} % para autoref
\usepackage{graphicx}
\usepackage{indentfirst}
\usepackage{textcomp}

\bibliography{bibliografia}


\begin{document}
%%
% Definições
%
\def\titulo{Projeto Final de Laboratórios de Sistemas Digitais}
\def\data{01 de Junho de 2023}
\def\autores{Olha Buts, André Correia}
\def\autorescontactos{(112920) o.buts@ua.pt, (87818) amcorreia@ua.pt}
\def\versao{VERSÃO 0.1}
\def\departamento{Dept. de Eletrónica, Telecomunicações e Informática}
\def\empresa{Universidade de Aveiro}
\def\logotipo{ua.pdf}
%
%%%%%% CAPA %%%%%%
%
\begin{titlepage}

\begin{center}
%
\vspace*{50mm}
%
{\Huge \titulo}\\ 
%
\vspace{10mm}
%
{\Large \empresa}\\
%
\vspace{10mm}
%
{\LARGE \autores}\\ 
%
\vspace{30mm}
%
\begin{figure}[h]
\center
\includegraphics{\logotipo}
\end{figure}
%
\vspace{30mm}
\end{center}
%
\begin{flushright}
\versao
\end{flushright}
\end{titlepage}

%%  Página de Título %%
\title{%
{\Huge\textbf{\titulo}}\\
{\Large \departamento\\ \empresa}
}
%
\author{%
    \autores \\
    \autorescontactos
}
%
\date{\today}
%
\maketitle

\pagenumbering{roman}

%%%%%% RESUMO %%%%%%
\begin{abstract}
popopopopopoqewp
\end{abstract}

%%%%%% Agradecimentos %%%%%%
% Segundo glisc deveria aparecer após conclusão...
%\renewcommand{\abstractname}{Agradecimentos}
%\begin{abstract}
%Eventuais agradecimentos.
%Comentar bloco caso não existam agradecimentos a fazer.
%\end{abstract}


\tableofcontents
% \listoftables     % descomentar se necessário
% \listoffigures    % descomentar se necessário


%%%%%%%%%%%%%%%%%%%%%%%%%%%%%%%
\clearpage
\pagenumbering{arabic}

%%%%%%%%%%%%%%%%%%%%%%%%%%%%%%%%
\chapter{Introdução}
\label{chap.introducao}
Os alunos da Unidade Curricular de \textbf{L}aboratórios de \textbf{S}istemas \textbf{D}igitais (\textit{LSD}, código 40333) da \textbf{L}icenciatura em \textbf{E}ngenharia de \textbf{C}omputadores e \textbf{I}nformática (\textit{LECI}, código 8316) da \textbf{U}niversidade de \textbf{A}veiro (\textit{UA}) foram propostos ao desenvolvimento de um Projeto Final que contempla três componentes: desenvolvimento do sistema digital de uma Máquina Automática de Fazer Pão (Projeto Número 8, Versão 2), criação de um relatório do desenvolvimento anteriormente referido, e a defesa do projeto perante um Juri.
\\\\
O sistema digital da Máquina Automática de Fazer Pão deve ser modelado em \textit{\textbf{V}ery High Speed Integrated Circuits \textbf{H}ardware \textbf{D}escription \textbf{L}anguage} (\textit{VHSIC-HDL}, ou \textit{VHDL}) e testado numa \textbf{F}ield-\textbf{P}rogrammable \textbf{G}ate \textbf{A}rray (\textit{FPGA}). Neste sentido, a máquina desenvolvida apresenta dois modos de operação principal: Fazer Pão Caseiro (Modo 1), ou Fazer Pão Rústico (Modo 2). Apesar de cada um destes modos ser caracterizado por diferentes parâmetros temporáis, ambos partilham a mesma \textit{pipeline}, ou procedimento de 'fazer pão'  (o amassar da massa, o descanso da massa para levedar, e a cozedura no final).
\\\\
Relativamente ao documento, este apresenta o relatório do desenvolvimento do sistema digital da Máquina Automática de Fazer Pão (Versão 2 do Projeto 8) de acordo com as competências adquiridas na Unidade Curricular de LSD.
Neste sentido, o documento divide-se em quatro componentes, sendo estas a arquitetura do sistema digital (descrição conceptual do sistema), a implementação efetuada para a anterior arquitetura (representação gráfica do sistema digital), os métodos de validação usados (simulações efetuadas sobre a implementação da arquitetura), e por fim, um manual de utilizador da máquina como um todo (em ambiente de desenvolvimento através de uma FPGA).

\chapter{Desenvolvimento do Sistema Digital}
\label{chap.desenvolvimentoSistemaDigital}
O desenvolvimento e implementação do sistema digital desta máquina passa por três fases: arquitetura (desenho lógico de todo o funcionamento do sistema), implementação (em VHDL, usando o programa Intel{\textsuperscript{\tiny{\textregistered}}} Quartus{\textsuperscript{\tiny{\textregistered}}} Prime) e uma posterior validação (testes via \textit{testbenches} em VHDL e via uso normal, numa ótica de utilizador).


\section{Arquitetura do Sistema}
Neste capítulo aborda-se a estrutura do sistema digital através de uma descrição conceptual da lógica que gerou o produto final (em ambiente de desenvolvimento por via de FPGA). Neste sentido, a arquitetura aplicada neste projeto divide-se em duas zonas que estão intrinsecamente interligadas:
\begin{itemize}
	\item Zona de controlo do sistema, responsável pelos inputs -- por exemplo:
	\begin{itemize}
		\item Conjunto (físico / \textit{hardware}) de \textit{keys}.
		\item Conjunto de \textit{switches}.
		\item Comportamento (lógico) de \textit{Start}/\textit{Stop}.
		\item Comportamento de \textit{reset} do sistema.
	\end{itemize}
	\item Zona de controlo do procedimento de 'fazer pão', responsável pelo output, e por toda a funcionalidade, da máquina (como um todo) -- por exemplo:
	\begin{itemize}
		\item Comportamento de cada etapa do processo de amassar, levedar e cozer o pão.
		\item Output do estado atual da máquina, assim como de informações ao utilizador, através de componentes físicos tais como o \textit{\textbf{L}iquid-\textbf{C}rystal \textbf{D}isplay} (LCD) e os \textit{7-Segment Displays} da FPGA.
	\end{itemize}
\end{itemize}
Estas duas zonas de controlo são ambas compostas por elementos lógicos \textit{standard}, assim como por controladores -- \textbf{M}áquinas de \textbf{E}stados \textbf{F}initos (MEF, ou FSM em Inglês). Estas FSM caracterizam-se por serem comunicantes, o que possibilita comportamentos que interligam simultâneamente a lógica das duas zonas de controlo -- destacando-se o caso da possibilidade de adicionar um tempo extra de atraso (antes de iniciar o procedimento de 'fazer pão'), ou a possibilidade de adicionar um tempo de cozedura extra (no final do processo de 'fazer pão').
\\\\
Assim, 
















\subsection{Utilização de acrónimos}
Esta é a primeira invocação do acrónimo \ac{ua}.
E esta é a segunda \ac{ua}.

Outra referência à \ac{leci}.

\subsection{Referências bibliográficas}
Informação relativa à estrutura formal de um relatório pode ser obtida
na página do \ac{glisc}\cite{glisc}.

\section{Implementação do Sistema}

\subsection{Utilização de acrónimos}
Esta é a primeira invocação do acrónimo \ac{ua}.
E esta é a segunda \ac{ua}.

\section{Validação do Sistema}

\subsection{Utilização de acrónimos}
Esta é a primeira invocação do acrónimo \ac{ua}.
E esta é a segunda \ac{ua}.

\chapter{Manual de Utilizador}
\label{chap.manualUtilizador}
Descreve os resultados obtidos.

\chapter{Conclusões}
\label{chap.conclusao}
A arquitetura da Máquina Automática de Fazer Pão e a posterior implementação através de Máquinas de Estados Finitos Comunicantes, e toda a envolvente lógica digital, demonstrou atingir um nível de complexidade que necessita de procedimentos de desenvolvimento bem definidos desde o início do projeto.
\\\\
Deste modo, é de importância realçar a necessidade de estratégias de desenvolvimento faseadas e de mecanismos de controlo, tais como versões de projeto, assim como metodologia em todas as etapas do projeto.
\\\\
Contudo --------


\chapter*{Contribuições dos autores}
Resumir aqui o que cada autor fez no trabalho.
Usar abreviaturas para identificar os autores,
por exemplo AS para António Silva.

\vspace{10pt}
\textbf{Indicar a percentagem de contribuição de cada autor.}\\


\autores : 50\%, 50\%\\

%\textbf{Repositório GitHub:} labi2023gN

%%%%%%%%%%%%%%%%%%%%%%%%%%%%%%%%%
%\chapter*{Acrónimos}
%\begin{acronym}
%\acro{ua}[UA]{Universidade de Aveiro}
%\acro{leci}[LECI]{Licenciatura em Engenharia de Computadores e Informática}
%\acro{glisc}[GLISC]{Grey Literature International Steering Committee}
%\acro{lsd}[LSD]{Laboratórios de Sistemas Digitais}
%\acro{vhdl}[VHDL]{Very \textit{High Speed Integrated Circuits} Hardware Description Language}
%\end{acronym}


%%%%%%%%%%%%%%%%%%%%%%%%%%%%%%%%%
\printbibliography

\end{document}
